\chapter{更新網站步驟}
\renewcommand{\baselinestretch}{10.0} %設定行距
\pagenumbering{arabic} %設定頁號阿拉伯數字
\setcounter{page}{1}  %設定頁數
\fontsize{14pt}{2.5pt}\sectionef
\section{詳細步驟說明}
機器學習與各領域結合的應用越來越廣泛,在機電系統採用強化學習是為了讓機電系統的控制達到最佳化。本專題以實體的冰球機,之機電系統作為訓練模型,將實體機器轉移到虛擬環境,進行模擬,為了找到適合的訓練參數,因此將模型簡化後再進行測試各種參數的優劣,透過不斷的訓練來得到一個優化過的對打系統。\\
\
1.個人的 fork 倉儲點選 sync fork 

2.打開資料夾 

3.點選 start_ipv6.bat 

4.輸入cd 資料夾名稱 並且git pull 

5.cms 

6.貼上網址進行動態網頁編輯 

7.acp 

8.點選個人 fork 倉儲 Open pull request 

9.回到整組倉儲 merge pull request
