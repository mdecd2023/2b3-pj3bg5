\renewcommand{\baselinestretch}{1.5} %設定行距
\pagenumbering{roman} %設定頁數為羅馬數字
\clearpage  %設定頁數開始編譯
\sectionef
\addcontentsline{toc}{chapter}{摘~~~要} %將摘要加入目錄
\begin{center}
\LARGE\textbf{摘~~要}\\
\end{center}
\begin{flushleft}
\fontsize{14pt}{20pt}\sectionef\hspace{12pt}\quad 此專題的分組需要改用 Python 的 zmqRemoteAPI 來進行控制。需要完成以下任務:使用 zmqRemoteAPI 控制八台雙輪車在足球場中 對戰。此外,還需要設計一個能在雙方瀏覽器中進行計分的系統。每個分組完成後,需要將相關議題、設計操作系統置於各組網站,並完成線上簡報和分組 PDF 報告書。\\[12pt]

\fontsize{14pt}{20pt}\sectionef\hspace{12pt}\quad 關於 zmqRemoteAPI:zmqRemoteAPI 是一個用於 Python 和 CoppeliaSim 之間通訊的工 具。下面是使用 zmqRemoteAPI 相對於其他 Python 和 CoppeliaSim 通訊方法的一些優點和缺點: 
\begin{enumerate}
\item 優點:快速高效、使用方便、跨平台兼容性\\
\item 缺點:與傳統的 Python remoteAPI 相比沒有缺點。\\[12pt]
\end{enumerate}

\fontsize{14pt}{20pt}\sectionef\hspace{12pt}\quad 使用 zmq Remote API 可以在 Python 中編寫代碼來控制 CoppeliaSim 中的 Bubblerob 模型。並且可以使用 Python 程式碼控制 Bubblerob 的移動和行為,並觀察結果。zmq Remote API 提供了一種高效且跨平台的方式,使 Python 和 CoppeliaSim 之間可以進行快速的通訊和交互。這使得您可以通過編寫 Python 程式碼來操縱和控制 Bubblerob 模型,並進行各種控制實驗。
\end{flushleft}
\newpage
%=--------------------Abstract----------------------=%
\renewcommand{\baselinestretch}{1.5} %設定行距
\addcontentsline{toc}{chapter}{Abstract} %將摘要加入目錄
\begin{center}
\LARGE\textbf\sectionef{Abstract}\\
\begin{flushleft}
\fontsize{14pt}{16pt}\sectionef\hspace{12pt}\quad The group project requires a switch to Python's zmqRemoteAPI for control. The following tasks need to be accomplished: using zmqRemoteAPI to control eight two-wheeled vehicles for a soccer match on a field. Additionally, a scoring system needs to be designed that allows scoring in both teams' browsers. After each group completes their tasks, they are required to document relevant issues, design the operational system on their respective group websites, and finalize an online presentation and a group PDF report.\\[12pt]

\fontsize{14pt}{16pt}\sectionef\hspace{12pt}\quad Regarding zmqRemoteAPI: zmqRemoteAPI is a communication tool used between Python and CoppeliaSim. Here are some advantages and disadvantages of using zmqRemoteAPI compared to other Python and CoppeliaSim communication methods:\\
\begin{enumerate}
\item Advantages:\\

Fast and efficient、Ease of use、Cross-platform compatibility.\\
\item Disadvantages:\\

There are no specific disadvantages of zmqRemoteAPI compared to traditional Python remoteAPI.\\[12pt]
\end{enumerate}

\fontsize{14pt}{16pt}\sectionef\hspace{12pt}\quad Using the zmq Remote API, you can write code in Python to control the Bubblerob model in CoppeliaSim. With Python code, you can control the movement and behavior of Bubblerob and observe the results. The zmq Remote API provides an efficient and cross-platform method for fast communication and interaction between Python and CoppeliaSim. This allows you to manipulate and control the Bubblerob model by writing Python code and conduct various control experiments.
\end{flushleft}
\begin{center}
\fontsize{14pt}{16pt}\selectfont\sectionef Keyword:  nerual network、reinforcement learning、 CoppeliaSim、OpenAI Gym
\end{center}
